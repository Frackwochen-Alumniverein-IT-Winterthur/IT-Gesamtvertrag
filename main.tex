\documentclass[fontsize=12pt,parskip=half]{scrartcl}

\usepackage[margin=2cm]{geometry}
\usepackage[utf8]{inputenc}
\usepackage[T1]{fontenc}
\usepackage[ngerman]{babel}
\usepackage{lmodern}
\usepackage[juratotoc]{contract}
\usepackage{color}
\usepackage{hyperref}

\makeatletter 
\renewcommand*{\parformat}{% 
  \global\hangindent 2em 
  \makebox[2em][l]{(\thepar)\hfill}%
}  
\makeatother
\renewcommand*{\parformatseparation}{}

\begin{document}

\title{Bartvertrag - 100 Jahre Barttradition \(DRAFT\)}
\maketitle


\section*{Präambel}
Vor 100 Jahren haben die Studenten des Technikums eine Tradition gegründet, bei der für 100 Tagen die angehenden Absolventen die Gesichtsbeharung nicht kürzen.
Der staatliche Bart dient als Eintritszeichen in die Gemeinschaft der Techniker und das volle Berufsleben und alle dessen Pflichten und Rechte.
Nach 100 Jahren ist es an der Zeit die 100 Tägige Tradition zu eneuern und zeitgemäss anzupassen.

\appendix % Trick to number Sections with Upper Case Letters

\section{Übersicht}
\begin{contract}

  \Clause[title={Zweck}]
  Der Vertrag regelt die Pflichten und Rechte der Unterzeichner/in, ab hier ``Vor-Würdige/in'', gegenüber ihre Komilitonen, der Würdigengemeinschaft und ihre IT-Vorfahren.

  \Clause[title={Gültigkeit}]\label{gueltigkeit}
  Der Vetrag tritt in Kraft, sobald ein/e Vor-Würdiger/in an der Letzten Rasur am 26. März 2025 unterschrieben hat und läuft, wenn nicht anders angegeben bis zum BartAb
  an der Nacht der Technik am 04. Juli 2025, 100 Tage später.

  Der Vetrag steht rechtlich über die Genfer Konventionen und insbesondere auch über das Allgemeine Zoll- und Handelsabkommen (GATT) der Internationalen Handelsorganisation (WTO),
  aber unter der schweizerischen Verfassung.

  \Clause[title={Struktur}]
  Der volgende Vetrag ist in vier Sektionen gegliedert, wobei die erste Sektion die ``Historikerverantwortung'' und die zweite Sektion die ``Bartverantwortung'' regeln. Diese Sektionen
  soorgen für den erhalt der Barttradition des IT-Studiengangs und sollen langfristig bis auf jährliche Datumsanpassungen stabil bleiben. \label{struktur.base}

  Die dritte Sektion regelt die ``Ersatzverantwortungen'' und erlaubt Klassen spezifische Ersatzverantwortungen zur Bartverantwortung zu definieren. Diese Sektion darf jährlich angepasst werden,
  wobei \refS{ersatz} unverändert bleibt.\label{struktur.ersatz}

  Die Sätze \refS{struktur.base}, \refS{struktur.ersatz} und \refS{struktur.schutz} dürfen nicht angepasst werden.\label{struktur.schutz}

  Die vierte Sektion beinhaltet die Unterschriften aller Vor-Würdigen, welche den Vertrag anerkennen und sich dessen Vorgaben verpflichten.

  \Clause[title={Rollen}]
  In diesem Artikel werden die Rollen enumeriert, die in diesem Vertrag vorkommen. Die genauen Aufgaben und Pflichten werden in den weiteren Artikeln definiert.

  \SubClause[title={Bartvögte}]
  Die Bartvögte, welche pro Klasse zwei ernannt werden, sind anerkannte Vor-Würdige, welche das stetige Einhalten dieses Vertrages überwachen.
  Als Ausgewählte gelten:\\[8ex]% adds space between the two sets of signatures
  \parnumberfalse
  \noindent\begin{tabular}{ll}
    \makebox[6.5cm]{\hrulefill} & \makebox[6.5cm]{\hrulefill} \\
    Bartvögt 1 - Klasse         & Bartvögt 2 - Klasse         \\[8ex]
    \makebox[6.5cm]{\hrulefill} & \makebox[6.5cm]{\hrulefill} \\
    Bartvögt 3 - Klasse         & Bartvögt 4 - Klasse         \\[8ex]
    \makebox[6.5cm]{\hrulefill} & \makebox[6.5cm]{\hrulefill} \\
    Bartvögt 5 - Klasse         & Bartvögt 6 - Klasse         \\[8ex]
    \makebox[6.5cm]{\hrulefill} & \makebox[6.5cm]{\hrulefill} \\
    Bartvögt 7 - Klasse         & Bartvögt 8 - Klasse         \\
  \end{tabular}
  \parnumbertrue

  \pagebreak

  \SubClause[title={Oberst-Bartvögt}]
  Der Oberst-Bartvögt ist der/die oberste Verantwortliche für die Einhaltung des Vertrages und wird unter den Bartvögten gewählt.
  Der/die Inhaber/in des Amts ist die höchste Instanz und ist verantwortlich für die Koordination unter den Bartvögten.
  In ihre Rolle ist die Person befugt Aufgaben an den Bartvögten zu definieren und zu delegieren.\\[8ex]
  \parnumberfalse
  \noindent\begin{tabular}{l}
    \makebox[6.5cm]{\hrulefill} \\
    Oberst-Bartvögt: Name       \\
  \end{tabular}
  \parnumbertrue

  \SubClause[title={IT-Vorfahre}]
  Der/Die IT-Vorfahre/-in ist der erwählte IT-Alumnivereinmitglied, der als Kontakperson zur Oberst-Bartvögt dient und die Koordination
  zwischen den IT-Alumniverein und Bartvögten übernimmt.\\[8ex]

  \parnumberfalse
  \noindent\begin{tabular}{l}
    \makebox[6.5cm]{\hrulefill} \\
    IT-Vorfahre: Name           \\
  \end{tabular}
  \parnumbertrue
\end{contract}

\pagebreak
\section{Historikerverantwortung}
\begin{contract}

  \Clause[title={``C-Handbuch unserer Geschichte''}]
  Zu Ehren von den Titanen der Informatik Brian W. Kernighan \& Dennis M. Ritch, die selber prächtige Bärte trugen,
  wird zum 100-jährigen Jubiläums der Barttradition eine Kopie des ``The C Programming Language'' Handbuches im
  eineitigen Format erstellt. Diese Kopie wird als ``C-Handbuch unserer Geschichte'' betauft, denn sie wird einen
  Abdruck von jeden kommenden Würdigen enthalten.

  \Clause[title={Verteilung}]
  Die Seiten des ``C-Handbuch unserer Geschichte'' werden an der Woche der Letzen Rasur unter den Vor-Würdigen gleichmässig verteilt.
  Bis zum Start der folgende müssen all Vor-Würdigen ihre Seiten in ihrem Besitz haben. Neben den Seiten des ``C-Handbuch unserer Geschichte''
  erhalten die Vor-Würdigen auch eine Badgehülle im A6 Format mit Kette. Diese Hülle, von hier die Erkennungshülle, enthällt eine Kopie der
  Titelseite des ``The C Programming Language'' Handbuches. Verantwortlich für die Verteilung sind die Bartvögte.

  \SubClause[title={Verteilungssicherheitsmarge}]\label{H.verteilungsmarge}
  Der Oberst-Bartvögt ist ermächtigt, bei der Verteilung der Seiten des ``C-Handbuch unserer Geschichte'' eine Sicherheitsmarge zu berücksichtigen um allfällige nachträgliche Vor-Würdigen
  zu ermöglichen, siehe \refS{H.nachzug}.

  \Clause[title={Nachträgliche Vor-Würdige}]\label{H.nachzug}
  Während den ersten 20 Tagen der 100 Tag Tradition dürfen Un-Würdige, welche den Bartvertrag nicht unterzeichnet haben, nachträglich den Bartvertrag unterzeichnen und die Historikerverantwortung
  aufnehmen. Dies muss aber vom Oberst-Bartvögt und mindestens ein weiterer Bartvögt genehmigt werden. Die genehmigenden Bartvögte müssen eine proportionale Strafe für den verspäteten Eintritt ernennen,
  die der neue Vor-Würdige erfüllen muss. Wir schlagen vor einen halbseitigen Entschuldigungsbrief, der an allen Vor-Würdigen verteilt wird. Der nachträgliche Vor-Würdige erhält die Seiten des ``C-Handbuch unserer Geschichte''
  und die Erkennungshülle von seinem Bartvögt. Die Seiten stammen aus der Verteungssicherheitsmarge \refS{H.verteilungsmarge}.

  \Clause[title={Erkennungs-, Schutz-, Vorweis-, Wissensaufgabe}] \label{H.aufgaben}
  Die Vor-Würdigen verpflichten sich vier Aufgaben zu erfüllen.

  \SubClause[title={Erkennungsaufgabe}]\label{H.erkennungsaufgabe}
  Die Vor-Würdigen verpflichten sich, die Erkennungshülle (oder äquivalentes \refS{aequivalentes}) bei jedem Aufenthalt am Technikum zu tragen während der
  Barttraditionsdauer sichtbar zu tragen. Dies dient als Erkennungszeichen für die Vor-Würdigen.

  Alle Vor-Würdigen sind ermächtigt und ermutigt, das Erkennungszeichen zu kontrollieren und bei Missachtung der Erkennungsaufgabe ihren zugeteilten Bartvögt davon zu informieren.

  Es ist in der Ermächtigung der Bartvögte ein äquivalentes Erkennungszeichen bilateral mit Interessenten zu vereinbaren.\label{aequivalentes}

  Wird ein/e Vor-Würdige/r von einer Person von ausserhalb der Würdigengemeinschaft, von hier als Zivilist/-in, auf ihr Erkennungszeichen angesprochen, so muss Sie eine Kombination
  der \refS{H.vorweisaufgabe} und \refS{H.wissensaufgabe} erfüllen und die Barttradition und Historikerverantwortung \refS{H.verantwortung} erklären.

  \SubClause[title={Schutzaufgabe}]\label{H.schutzaufgabe}
  Die Vor-Würdigen verpflichten sich, die Seiten des ``C-Handbuch unserer Geschichte'' vor jedem Schaden zu schützen.

  \SubClause[title={Vorweisaufgabe}]\label{H.vorweisaufgabe}
  Die Vor-Würdigen verpflichten sich, die Seiten des ``C-Handbuch unserer Geschichte'' auf Anfrage vorzuweisen. Dies kann durch Vorweisen der originalen
  Seiten oder einer digitalen Kopie erfolgen.

  \SubClause[title={Wissensaufgabe}]\label{H.wissensaufgabe}
  Die Vor-Würdigen verpflichten sich, den Inhalt ihrer Seiten zu studieren und zu verstehen. Sie müssen in der Lage sein, die Informationen
  auf Anfrage zusammenzufassen und zu erklären. Ist die Anfragende Person zufrieden mit der Antwort, so wird Sie vom Vor-Würdigen gebeten
  ihre Unterschrift auf die Rückseite der Titelseite der Erkennungshülle (und NICHT eine Seit des ``C-Handbuch unserer Geschichte'') und einem
  Timestamp der Interaktion zu hinterlassen.

  Die Wissensaufgabe gilt nur als erfüllt, wenn mindestens eine Unterschrift gesammelt wurde. Die Anfragende Person muss nicht Teil der Würdigengemeinschaft sein.

  Die Bartvögte sind ermutigt, die Vor-Würdigen unter sich zu verteilen und zu prüfen.

  Der Vor-Würdiger mit den meisten einzigartigen Unterschriften wird mit einem Getränkegutschein von einem noch zu definierendem Wert, oder äquivalentes
  belohnt. Haben mehrere Vor-Würdige die gleiche Anzahl an Unterschriften, so wird der Gutschein unter diesen geteilt.

  \SubClause[title={Dauer}]
  Die Schutzaufgabe \refS{H.schutzaufgabe} gilt von Annahme der Seiten des ``C-Handbuch unserer Geschichte'' bis zur Einsammlung \refS{H.einsammlung}.

  Die Erkennungs- \refS{H.erkennungsaufgabe}, Vorweis- \refS{H.vorweisaufgabe} und Wissensaufgabe \refS{H.wissensaufgabe} gelten von Annahme der Seiten
  des ``C-Handbuch unserer Geschichte'' bis zum BartAb an der Nacht der Technik am 04. Juli 2025.

  \Clause[title={Historikerverantwortung}]\label{H.verantwortung}
  Die in \refS{H.aufgaben} aufgezählten Aufgaben sind die Pflichten aller Vor-Würdigen und sind nicht abtragbar und definieren die Historikerverantwortung.
  Bei erfolgreicher Erfüllung der Historikerverantwortung bis zum Tag der Diplomfeier wird der Vor-Würdiger offiziel als Würdiger anerkannt und darf sich
  als Mitglied der Würdigengemeinschaft bezeichnen. Die Annerkennung wird durch eine Unterschrift des Würdigen auf eine seiner Seiten des ``C-Handbuch unserer Geschichte''
  an der Diplomfeier vor der Einsammlung \refS{H.einsammlung} bestätigt. Dafür stehene die Ränder und Rückseite der Seiten zur Verfügung. Die Würdigen sind
  gebeten auf die nächsten Generationen zu achten und den verwendeten Platz für ihre Unterschrift nicht zu übertreiben.

  \Clause[title={Un-Würdige}]
  Als Unwürdig gelten alle Klassenmitglieder, welche den Bartvertrag nicht unterzeichnet haben. Diese werden öffentlich geächtet und gehen der Geschichte verloren. In 100 Jahren
  bleibt das "C-Handbuch unserer Geschichte" weiter bestehen aber Un-Würdige nicht.

  Bei Missachtung der Historikerverantwortung \refS{H.verantwortung} wird der Vor-Würdiger von der Würdigengemeinschaft ausgeschlossen und ist für die Ewigkeit als
  Un-Würdiger gestuft.

  \Clause[title={Anti-Würdige}]
  Die Missachtung der Schutzaufgabe \refS{H.schutzaufgabe} gilt als schwerwiegend, da es die Integrität des ``C-Handbuch unserer Geschichte'' und die
  damit verbundene Kontinuität unserer Gemeinde gefährdet. Bei Verlust oder groben Schaden an den Seiten des ``C-Handbuch unserer Geschichte'' wird der Vor-Würdiger sogar als
  Anti-Würdiger gestuft und die Schmach wird auf seine ganzen Blutlinie übertragen.

  \Clause[title={Einsammlung}]\label{H.einsammlung}
  Die Seiten des ``C-Handbuch unserer Geschichte'' werden an der Diplomfeier von einem designierten Bartvögt eingesammelt und an den IT-Vorfahre übergeben.

  Da für die Campuse der Zürcher Hochschule für Angewandte Wissenschaften (ZHAW) getrennte Diplomfeier stattfinden, wird ein Würdiger erwählt, der die gesammelten
  Seiten einer Diplomfeier an den Bartvögten der nächsten Diplomfeier übergibt. Dieser Würdiger erhält den ehrenvollen Titel ``Würdiger der Vereinigung der Diplomfeiern'' und darf
  diesen Titel mit Jahr unter seine Unterschrift im ``C-Handbuch unserer Geschichte'' notieren.

  Nach der Vereinigung des ``C-Handbuch unserer Geschichte'' wird ein Würdiger ernannt der es an den IT-Alumniverein übergeben, welcher die Archivistenaufgabe \refS{H.archivistenaufgabe} übernimmt.

  Die Erkennungshüllen werden spätestens an der Diplomfeier abgegeben und an die Bartvögte übergeben und and den IT-Vorfahre weitergeleitet.

  Die unterschriebene Titelseite der Erkennungshülle bleibt im Besitz des Würdigen und dient als Beweis der Erfüllung der Historikerverantwortung.

  \Clause[title={Archivistenaufgabe}]\label{H.archivistenaufgabe}
  Der IT-Alumniverein verpflichtet sich, das Original des ``C-Handbuch unserer Geschichte'' sowie auch digitalisierte Kopien zu archivieren und während der Barttraditionfreiezeit
  zu pflegen. Der IT-Vorfahre ist verantwortlich für die Koordination und die Übergabe an den Bartvögten der nächsten Generation.

  Ist das ``C-Handbuch unserer Geschichte'' mit Unterschriften der Würdigen voll, so wird es als ``C-Handbuches unserer Geschichte V1.0'' getauft und eine neue Kopie erstellt.

  Die Versionsnummer wird auf der Titelseite des ``C-Handbuch unserer Geschichte'' notiert und die alte Version wird im Archiv des IT-Alumnivereins aufbewahrt.

  Mit jeder neuen Version wird die Major-Version Nummer inkrementiert.

  \Clause[title={Fittnessklausel}]
  Wenn ein Vor-Würdiger dem Oberst-Bartvögt eine Sequenz von 13 Pistolsquats pro Bein vorführt, wird er/sie automatisch als Würdiger anerkannt und darf seine Unterschrift
  dem ``C-Handbuch unserer Geschichte'' mit dem Titel ``Würdiger der Fittness'' hinzufügen und seine Seiten an den Oberst-Bartvögt übergeben. Er/Sie ist von allen weiteren
  Aufgaben und Pflichten befreit.

\end{contract}

\pagebreak
\section{Bartverantwortung}
\begin{contract}
  \Clause[title={Bartverantwortung}]
  Zusätzlich zu der Historikerverantwortung \refS{H.verantwortung} dürfen Vor-Würdige auch die Bartverantwortung übernehmen. Die Bartverantwortung ist
  strickt optional.

  \Clause[title={Letzte Rasur}]\label{B.lastrasur}
  Jeder Vor-Würdige, das zusätzlich die Bartverantwortung übernehmen will, hat sich am in \refS{gueltigkeit} genannten Datum um 13:00 Uhr
  beim Fischermädchenbrunnen in der Steinberggasse einzufinden.
  Die Rasur findet gemeinsam statt.

  \SubClause[title={Durchführung der Rasur}]\label{B.durchfuehrung}
  Um die Tradition zu ehren, sind ausschliesslich Nassrasuren gestattet. Die Verwendung von
  elektronischen Gerätschaften und sonstiger Hexereien ist strengstens untersagt. Erforderlich für eine
  ehrenvolle Rasur ist die vollständige Entfernung der Gesichtsbehaarung, ausgenommen der
  Augenbrauen. Die Bartvögte nehmen sich der Begutachtung der Rasur jedes einzelnen Mitglieds
  an.
  \SubClause[title={Anschliessende Kehlenbefeuchtung}]
  Am Ende der Rasur sind alle Würdigen eingeladen, vor Ort ein kühles Bier einzunehmen.

  \SubClause[title={Rasurverzug}]
  Ist einem ehrenvollen Mitglied die Rasur zur vorgesehenen Zeit und Stelle nicht möglich,
  ist es diesem gestattet, die Rasur bis zum 31. März 2025 nachzuholen, sofern ein triftiger
  Grund vorgewiesen werden kann. Die Bartvögte entscheiden über die Zulässigkeit des Grundes
  und überprüfen die Rasur, wenn möglich persönlich oder alternativ durch einen
  Microsoft Teams-Call mit eingeschalteter Webcam.

  \SubClause[title={Glaubensklausel (Shamess Klausel)}]
  Sollte die Glattrasur aufgrund glaubenstechnischer Regelungen untersagt sein, meldet sich
  das betroffene ehrenvolle Vor-Würdige bei den Bartvögten, um die Rasur-Einschränkung zu
  besprechen. Sollten die Bartvögte mit der Mindestlänge an Bart einverstanden sein,
  kann das ehrenvolle Mitglied, abgesehen der eingeschränkten Rasuren, normal
  an der Bartverantwortung teilnehmen.

  \Clause[title={Bartvorschriften}]
  \SubClause[title={Definition Bart}]
  Als Bart wird eine ununterbrochene Linie mit körpereigenen Gesichtshaaren zwischen den beiden Ohren,
  welche über Kinn und Oberlippe verläuft und bis zum Halsansatz reicht, verstanden. Eine Bartaneignung
  durch Fremdhilfe wie Haartransplantation, Bartimitationen, tätowierte oder aufgemalte Bärte,
  Skalpe oder ähnliches sowie die Verwendung von Haarwuchsmittel sind untersagt und gelten als Vertragsbruch.

  \SubClause[title={Bartpflege}]\label{B.bartpflege}
  Die Würde des Bartes ist durch regelmässige Pflege in Ehren zu halten. Ungepflegte, verfilzte oder gar
  verlauste Bärte, sind der langen Tradition nicht würdig. Das Vernachlässigen der adäquaten Bartpflege
  wird von den Bartvögten geahndet. Vor-Würdige, die sich dessen schuldig gemacht haben, erwartet die
  Einnahme eines Rachenputzers \refS{B.rachenputzer} oder den Ausschank eines Bieres an zwei beliebige Vor-Würdige der Klasse.
  Zudem muss die würdige Person die regelmässige Pflege des Bartes innert 24 Stunden aufnehmen. Bei
  weiteren Verstössen erhöht sich die Anzahl der Rachenputzer oder Bierausschänke pro Verstoss um den
  Faktor zwei. Das Trimmen von Kontur und die Stutzung des Schnauzes, um eine einwandfreie
  Lebensmittelaufnahme zu gewährleisten ist erlaubt. Regelmässiges Waschen, Föhnen und Kämmen wird empfohlen.

  \SubClause[title={Rachenputzer}]\label{B.rachenputzer}
  Was genau als Rachenputzer gilt, soll den Bartvögten überlassen werden. Grundsätzlich versteht sich unter
  Rachenputzer ein Getränk alkoholischer oder nichtalkoholischer Natur. Ein Rachenputzer soll als Bestrafung fungieren,
  wobei der Wiedereingliederungsgedanke in die würdige Gesellschaft im Vordergrund stehen soll.
  Die Ehre des würdigen Vor-Würdigen darf darum nicht mehr als nötig verletzt werden.

  \SubClause[title={Schneiden}]
  Ab dem Zeitpunkt der letzten Rasur gemäss \refS{B.lastrasur} darf dem Bartwuchs nicht entgegengewirkt werden.
  Jedes Kürzen der Bartbehaarung, ausgenommen ist der Fall\refS{B.bartpflege}, ist untersagt und gilt als Bruch der Bartverantwortung.

  \Clause[title={Bartprüfung}]\label{B.bartpruefung}
  Die Bartverantwortung gilt als abgelegt, wenn die Bartprüfung bestanden wurde. Die Bartprüfung wird volgendermassen definiert:

  \SubClause[title={Durchführung}]
  Um die Würde der Bartträgerschaft endgültig zu festigen, müssen sich alle Vor-Würdigen, die die Bartverantwortung aufgenommen haben,
  einer Bartprüfung unterziehen, welche am Tag vor der Nacht der Technik stattfindet. Die Prüfung wird von den
  Bartvögten geleitet und durchgeführt. Der Durchführungsort ist ebenfalls der Fischermädchenbrunnen in der
  Steinberggasse.

  \SubClause[title={Handlung}]
  Traditionsgemäss steckt sich jede Person, die die Prüfung ablegen möchte, einen Bleistift in den Bart. Die
  Prüfung gilt als bestanden, wenn der Bleistift nach einem Sprung vom Brunnenrand auf den Boden noch an
  derselben Stelle sitzt. Die Prüfung darf dreimal wiederholt werden. Wird die Prüfung auch nach dem dritten Versuch nicht bestanden,
  so zählt dieses Mitglied zu den Gescheiterten \refS{B.gescheiterte}.

  \SubClause[title={Zustand des Bartes}]
  Zum Zeitpunkt der Prüfung muss sich der Bart in sauberem Zustand befinden. Klebrige Rückstände von Getränken,
  Speisen, Gel, Wachs oder Ähnlichem müssen vollständig entfernt worden sein.

  \SubClause[title={Prüfungsaufsicht}]
  Die Prüfungsaufsicht obliegt den Bartvögten.

  \SubClause[title={Prüfung der Bartvögte}]
  Um die Unabhängigkeit der Prüfungsbewertung zu gewährleisten, dürfen die Bartvögte ihre eigene Prüfung nicht selbst bewerten.
  Zu diesem Zweck werden aus der versammelten Gemeinschaft zwei Interimsvögte ausgewählt, die die Prüfung
  der Bartvögte beaufsichtigen. Diese werden nach dem Zufallsprinzip ausgewählt. Zur Auswahl dürfen Würfel,
  Schere-Stein-Papier, A zellä Böllä schelä oder Ähnliches zum Einsatz kommen. Die Bartvögte legen die
  Prüfung als erste ab.

  \Clause[title={BartAb}]
  Nach Ablegen der Bartprüfung und der Durchführung des Frackumzugs darf sich jeder Vor-Würdige, der die Bartverantwortung aufgenommen
  hat am Bart Ab Event rasieren lassen, wobei dieselben Bestimmungen des \refS{B.durchfuehrung} gültig sind.

  \Clause[title={Gescheiterte}]\label{B.gescheiterte}
  Als Gescheiterte gelten Vor-Würdige, welche trotz ihres guten Willens die Prüfung gemäss \refS{B.bartpruefung}
  aufgrund ihres spärlichen Bartwuchses nicht bestanden haben.

  \Clause[title={Titel}]\label{B.barttitel}
  Eine Erfüllung der Bartverantwortung gemäss \refS{B.bartpruefung} berechtigt den Vor-Würdigen bei Erfüllung der Historikerverantwortung \refS{H.verantwortung}
  und der Erhebung zum Würdigen, zum Titel ``Bartwürdiger''. Damit darf dessen Unterschrift im ``C-Handbuch unserer Geschichte'' mit dem Titel ``Bartwürdiger'' oder eine andere
  Beschmückung versehen werden.

\end{contract}

\pagebreak
\section{Ersatzverantwortungen}
\begin{contract}
  \Clause[title={Ersatzverantwortung}]\label{ersatz}
  Anstelle der Bartverantwortung dürfen einzelne Klassen eine Ersatzverantwortung definieren. Diese Verantwortungen und dessen Strafen werden klassenintern
  definiert und geregelt.

  Die Ersatzverantwortung befreit die Vor-Würdigen von der Historikerverantwortung \refS{H.verantwortung} NICHT.

  Die Ersatzverantwortung gilt als äquivalent zur Bartverantwortung und berechtigt bei Erfüllung einen äquivalenten Titel zu \refS{B.barttitel} bei der Unterzeichnung
  im ``C-Handbuch unserer Geschichte'' nach erfüllung der Historikerverantwortung \refS{H.verantwortung}. Der Titel muss im Vertrag definiert sein.

  \Clause[title={Klasse IT22a WIN}]
  Die Klasse IT22a WIN hat sich für eine Ersatzverantwortung entschieden und definiert diese in den folgenden Unterartikeln.

  \SubClause[title={Ersatzverantwortung}]\label{KlasseIT22aWIN.ersatz}
  Während der \refS{gueltigkeit} dargelegten Zeitspanne sind sie dazu verpflichtet, einen bizli zu grossen Zylinder-Hut auf dem Kopf zu tragen,
  sobald der Campus der ZHAW betreten wird. Dabei kann der Hut vom Würdigen selbst beschafft werden, solange das Model von den Bartvögten
  genehmigt wurde. Andernfalls organisieren die Bartvögte den Zylinder-Hut.

  Die Verantwortung trägt den Namen ``Zylinderverantwortung''.\label{KlasseIT22aWIN.verantwortung.name}

  \SubClause[title={Verstoss}]
  Werden Verantwortungsbrüche der Vor-Würdigen nach \refS{KlasseIT22aWIN.ersatz} gesichtet, dann sind sie an der Zylinderverantwortung
  \refS{KlasseIT22aWIN.verantwortung.name} gescheitert. Möchte eine Gescheiterte Person nach \refS{ersatz} trotzdem den Titel nach
  \refS{KlasseIT22aWIN.titel} verdienen, muss sie innert Jahresfrist, also bis zum 4. Juli 2026, eine Klassenzusammenkunft organisieren
  und einen adäquaten Biervorrat sicherstellen.

  Die versammelte Klasse muss intern mit dem IT-Alumniverein koordinieren, um Zugang zum ``C-Handbuch unserer Geschichte'' zu erhalten und die Unterschrift
  mit dem erworbenen Titel zu erweitern.

  \SubClause[title={Titel}]\label{KlasseIT22aWIN.titel}
  Die Vor-Würdigen der Klasse IT22a WIN, die die Ersatzverantwortung nach \refS{KlasseIT22aWIN.ersatz} erfüllen, verdienen den Titel ``Zylinderwürdiger''.

  \Clause[title={Klasse IT22a ZH}]
  Die Klasse IT22a ZH verzichtet auf eine Ersatzverantwortung.

  \Clause[title={Klasse IT21ta WIN}]
  Die Klasse IT21ta WIN gilt gesammt als Un-Würdig, da sie den Versuch eines Gesamtbartvertrages abgelehnt hat bzw. die Mehrheit kein Interesse an der Frackwoche hat.

  \Clause[title={Klasse IT21tb WIN}]
  Die Klasse IT21tb WIN verzichtet auf eine Ersatzverantwortung.
\end{contract}


\pagebreak
\section{Unterschriften}
\vspace{50pt}
\noindent\rule{7cm}{.4pt}\hfill\rule{7cm}{.4pt}\par
\noindent Unterschrift \hfill Unterschrift

\vspace{50pt}
\noindent\rule{7cm}{.4pt}\hfill\rule{7cm}{.4pt}\par
\noindent Unterschrift \hfill Unterschrift

\vspace{50pt}
\noindent\rule{7cm}{.4pt}\hfill\rule{7cm}{.4pt}\par
\noindent Unterschrift \hfill Unterschrift

\vspace{50pt}
\noindent\rule{7cm}{.4pt}\hfill\rule{7cm}{.4pt}\par
\noindent Unterschrift \hfill Unterschrift

\vspace{50pt}
\noindent\rule{7cm}{.4pt}\hfill\rule{7cm}{.4pt}\par
\noindent Unterschrift \hfill Unterschrift

\vspace{50pt}
\noindent\rule{7cm}{.4pt}\hfill\rule{7cm}{.4pt}\par
\noindent Unterschrift \hfill Unterschrift

\vspace{50pt}
\noindent\rule{7cm}{.4pt}\hfill\rule{7cm}{.4pt}\par
\noindent Unterschrift \hfill Unterschrift

\vspace{50pt}
\noindent\rule{7cm}{.4pt}\hfill\rule{7cm}{.4pt}\par
\noindent Unterschrift \hfill Unterschrift

\vspace{50pt}
\noindent\rule{7cm}{.4pt}\hfill\rule{7cm}{.4pt}\par
\noindent Unterschrift \hfill Unterschrift

\end{document}