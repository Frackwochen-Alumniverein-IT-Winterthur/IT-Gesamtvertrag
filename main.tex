\documentclass[parskip=half]{scrreprt}

\usepackage[margin=2cm]{geometry}
\usepackage[utf8]{inputenc}
\usepackage[T1]{fontenc}
\usepackage[ngerman]{babel}
\usepackage{lmodern}
\usepackage[juratotoc]{scrjura}
\usepackage{color}

\makeatletter 
\renewcommand*{\parformat}{% 
  \global\hangindent 2em 
  \makebox[2em][l]{(\thepar)\hfill}%
}  
\makeatother
\renewcommand*{\parformatseparation}{}

\begin{document}
\addchap{Bartvertrag}


\section*{Präambel}
Vor 100 Jahren haben die Studenten des Technikums eine Tradition gegründet, bei der für 100 Tagen die angehenden Absolventen die Gesichtsbeharung nicht kürzen.
Der staatliche Bart dient als Eintritszeichen in die Gemeinschaft der Techniker und das volle Berufsleben und alle dessen Pflichten und Rechte.
Nach 100 Jahren ist es an der Zeit die 100 Tägige Tradition zu eneuern und zeitgemäss anzupassen.

\begin{contract}
  \Clause{title={Zweck, Gültigkeit und Struktur}}

  \SubClause{title={Zweck}}
  Der Vertrag regelt die Pflichten und Rechte der Unterzeichner/in, ab hier ``Vor-Würdige/in'', gegenüber ihre Komilitonen, der Würdigengemeinschaft und ihre IT-Vorfahren.

  \SubClause{title={Gültigkeit}}
  Der Vetrag tritt in Kraft, sobald ein/e Vor-Würdiger/in an der letzten Rasur am 26. März 2025 unterschrieben hat und läuft, wenn nicht anders angegeben bis zum BartAb an der Nacht der Technik am 04. Juli 2025, 100 Tage später.

  Der Vetrag steht rechtlich über die Genfer Konventionen und insbesondere auch über das Allgemeine Zoll- und Handelsabkommen (GATT) der Internationalen Handelsorganisation (WTO) aber unter der schweizerischen Verfassung.

  \SubClause{title={Struktur}}
  Der Vetrag ist in drei Teilen gegliedert, wobei der erste Teil die ``Historikerverantwortung'' und der zweite Teil die ``Bartverantwortung'' regeln.
  Die Artikel *Insert Article Numbers hier* widmen sich der Historikerverantwortung und die Artikel *Insert Article Numbers hier* der Bartverantwortung.
  Im dritten Teil werden allfällige Klasseninterne ``Verantwortugen'' geregelt.

  \Clause{title={Rollen}}
  In diesem Artikel werden die Rollen enumeriert, die in diesem Vertrag vorkommen. Die genauen Aufgaben und Pflichten werden in den weiteren Artikeln definiert.

  \SubClause{title={Bartvögte}}
  Die Bartvögte, welche durch die Gesetzschreiber ernannt wurden, sind anerkannte Mitglieder, welche das stetige Einhalten dieses Vertrages überwachen.
  Als Ausgewählte gelten:\\[8ex]% adds space between the two sets of signatures
  \parnumberfalse
  \noindent\begin{tabular}{ll}
    \makebox[6.5cm]{\hrulefill} & \makebox[6.5cm]{\hrulefill} \\
    Bartvögt 1                  & Bartvögt 2                  \\[8ex]
    \makebox[6.5cm]{\hrulefill} & \makebox[6.5cm]{\hrulefill} \\
    Bartvögt 3                  & Bartvögt 4                  \\[8ex]
    \makebox[6.5cm]{\hrulefill} & \makebox[6.5cm]{\hrulefill} \\
    Bartvögt 5                  & Bartvögt 6                  \\
  \end{tabular}
  \parnumbertrue

  \SubClause{title={Oberst-Bartvögt}}
  Der Oberst-Bartvögt ist der/die oberste Verantwortliche für die Einhaltung des Vertrages und wird unter den Bartvögten gewählt.
  Der/die Inhaber/in des Amts ist die höchste Instanz und ist verantwortlich für die Koordination unter den Bartvögten.
  In ihre Rolle ist die Person befugt Aufgaben and den Bartvögten zu definieren und zu delegieren.\\[8ex]
  \parnumberfalse
  \noindent\begin{tabular}{l}
    \makebox[6.5cm]{\hrulefill} \\
    Oberst-Bartvögt             \\[8ex]
  \end{tabular}
  \parnumbertrue

  \SubClause{title={IT-Vorfahre}}
  Der/Die IT-Vorfahre ist der erwählte IT-Alumnivereinmitglied, der als Kontakperson zur Oberst-Bartvögt dient und die Koordination
  zwischen den IT-Alumniverein und Bartvögten übernimmt.
  \parnumberfalse
  \noindent\begin{tabular}{l}
    \makebox[6.5cm]{\hrulefill} \\
    IT-Vorfahre                 \\[8ex]
  \end{tabular}
  \parnumbertrue

  \Clause{title={Historikerverantwortung}}


\end{contract}

\vspace{50pt}
\noindent\rule{7cm}{.4pt}\hfill\rule{7cm}{.4pt}\par
\noindent Datum, Ort \hfill Unterschrift Verkäufer

\vspace{50pt}
\noindent\rule{7cm}{.4pt}\hfill\rule{7cm}{.4pt}\par
\noindent Datum, Ort \hfill Unterschrift Käufer

\end{document}