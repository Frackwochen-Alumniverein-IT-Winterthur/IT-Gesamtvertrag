\documentclass[fontsize=12pt,parskip=half]{scrartcl}

\usepackage[margin=2cm]{geometry}
\usepackage[utf8]{inputenc}
\usepackage[T1]{fontenc}
\usepackage[nswissgerman]{babel}
\usepackage{lmodern}
\usepackage[juratotoc]{contract}
\usepackage{color}
\usepackage{hyperref}
\usepackage{enumitem}

\addtokomafont{disposition}{\rmfamily}
\addtokomafont{contract.Clause}{\rmfamily}

\makeatletter 
\renewcommand*{\parformat}{% 
  \global\hangindent 2em 
  \makebox[2em][l]{(\thepar)\hfill}%
}  
\makeatother
\renewcommand*{\parformatseparation}{}
\renewcommand{\thesection}{\Roman{section}} 

\begin{document}

\title{Bartvertrag - 100 Jahre Barttradition}
\maketitle


\section*{Präambel}
Vor 100 Jahren haben die Studierenden des Technikums eine Tradition gegründet, bei der für 100 Tage die angehenden Absolventen die Gesichtsbeharung nicht kürzen.
Der staatliche Bart dient als Eintritszeichen in die Gemeinschaft der Techniker und das volle Berufsleben und alle dessen Pflichten und Rechte.
Nach 100 Jahren ist es an der Zeit die 100-tägige Tradition zu eneuern und zeitgemäss anzupassen.

\section{Übersicht}
\begin{contract}

  \Clause[title={Zweck}]
  Der Vertrag regelt die Pflichten und Rechte der Unterzeichnenden, ab hier "<Vor-Würdige">, gegenüber ihren Komiliton:innen, der Würdigengemeinschaft und ihren IT-Vorfahren.

  \Clause[title={Gültigkeit}]\label{gueltigkeit}
  Der Vetrag tritt in Kraft, sobald ein:e Vor-Würdige:r an der Letzten Rasur am 26. März 2025 unterschrieben hat und läuft, wenn nicht anders angegeben bis zum BartAb
  an der Nacht der Technik am 04. Juli 2025, 100 Tage später.

  Der Vetrag steht rechtlich über den Genfer Konventionen und insbesondere auch über dem Allgemeinen Zoll- und Handelsabkommen (GATT) der Welthandelsorganisation (WTO),
  aber unter der Schweizerischen Verfassung.

  \Clause[title={Struktur}]
  Der vorliegende Vetrag ist in vier Sektionen gegliedert, wobei die erste Sektion die "<Historikerverantwortung"> und die zweite Sektion die "<Bartverantwortung"> regeln. Diese Sektionen
  sorgen für den Erhalt der Barttradition des IT-Studiengangs und sollen langfristig bis auf jährliche Datumsanpassungen stabil bleiben. \label{struktur.base}

  Die dritte Sektion regelt die "<Ersatzverantwortungen"> und erlaubt Klassen spezifische Ersatzverantwortungen zur Bartverantwortung zu definieren. Diese Sektion darf jährlich angepasst werden,
  wobei \refS{ersatz} unverändert bleibt.\label{struktur.ersatz}

  Die Sätze \refS{struktur.base}, \refS{struktur.ersatz} und \refS{struktur.schutz} dürfen nicht angepasst werden.\label{struktur.schutz}

  Die vierte Sektion beinhaltet die Unterschriften aller Vor-Würdigen, welche den Vertrag anerkennen und sich dessen Vorgaben verpflichten.

  \Clause[title={Rollen}]
  In diesem Artikel werden die Rollen definiert, die in diesem Vertrag vorkommen. Die genauen Aufgaben und Pflichten werden in den weiteren Artikeln erläutert.

  \SubClause[title={Bartvögte}]
  Pro Klasse werden zwei Bartvögte ernannt. Dies sind anerkannte Vor-Würdige, welche das stetige Einhalten dieses Vertrages überwachen.
  Als Ausgewählte gelten:\\[8ex]% adds space between the two sets of signatures
  \parnumberfalse
  \noindent\begin{tabular}{ll}
    \makebox[6.5cm]{\hrulefill}    & \makebox[6.5cm]{\hrulefill} \\
    Elhayawan, Karim - IT21 tb WIN & Bartvogt 2 - IT21 tb WIN    \\[8ex]
    \makebox[6.5cm]{\hrulefill}    & \makebox[6.5cm]{\hrulefill} \\
    Ueltschi, Damian - IT22 a ZH   & Willi, Leandra - IT22 a ZH  \\[8ex]
  \end{tabular}
  \parnumbertrue

  \pagebreak

  \SubClause[title={Oberst-Bartvogt}]
  Der Oberst-Bartvogt ist der/die oberste Verantwortliche für die Einhaltung des Vertrages und wird unter den Bartvögten gewählt.
  Der/die Inhaber/in des Amts ist die höchste Instanz und ist verantwortlich für die Koordination unter den Bartvögten.
  In ihrer Rolle ist die Person befugt, Aufgaben an die Bartvögte zu delegieren.\\[8ex]
  \parnumberfalse
  \noindent\begin{tabular}{l}
    \makebox[6.5cm]{\hrulefill}      \\
    Oberst-Bartvogt: Karim Elhayawan \\
  \end{tabular}
  \parnumbertrue

  \SubClause[title={IT-Vorfahre}]
  Der/Die IT-Vorfahre:in ist das erwählte IT-Alumnivereinmitglied, das als Kontakperson zum Oberst-Bartvogt dient und die Koordination
  zwischen dem ``Frackwochen- und Alumniverein Informatik Winterthur'' (nachfolgend FAI) und den Bartvögten übernimmt.\\[8ex]

  \parnumberfalse
  \noindent\begin{tabular}{l}
    \makebox[6.5cm]{\hrulefill} \\
    IT-Vorfahre: Name           \\
  \end{tabular}
  \parnumbertrue
\end{contract}

\pagebreak
\section{Historikerverantwortung}
\begin{contract}

  \Clause[title={"<C-Handbuch unserer Geschichte">}]
  Zu Ehren von den Titanen der Informatik Brian W. Kernighan \& Dennis M. Ritch, die selber prächtige Bärte trugen,
  wird zum 100-jährigen Jubiläum der Barttradition eine Kopie des "<The C Programming Language"> Handbuches im
  einseitigen Format erstellt. Diese Kopie wird als "<C-Handbuch unserer Geschichte"> betauft, denn sie wird einen
  Eintrag von allen zukünftigen Würdigen enthalten.

  \Clause[title={Verteilung}]
  Die Seiten des "<C-Handbuch unserer Geschichte"> werden an der Woche der Letzen Rasur unter den Vor-Würdigen grundsätzlich, siehe \refS{H.verteilungsmarge}, gleichmässig verteilt.
  Bis zum Start der folgenden Woche müssen all Vor-Würdigen ihre Seiten in ihrem Besitz haben.
  Neben den Seiten des "<C-Handbuch unserer Geschichte">
  erhalten die Vor-Würdigen auch eine Badgehülle im A6-Format mit Kette. Diese Hülle, ab hier die ``Erkennungshülle'' genannt, enthällt eine Kopie der
  Titelseite des "<The C Programming Language"> Handbuches. Verantwortlich für die Verteilung sind die Bartvögte.

  \SubClause[title={Verteilungssicherheitsmarge}]\label{H.verteilungsmarge}
  Der Oberst-Bartvogt ist ermächtigt, bei der Verteilung der Seiten des "<C-Handbuch unserer Geschichte"> eine Sicherheitsmarge zu berücksichtigen um allfällige nachträgliche Vor-Würdigen
  die Teilnahme zu ermöglichen, siehe \refS{H.nachzug}.

  \Clause[title={Nachträgliche Vor-Würdige}]\label{H.nachzug}
  Während der ersten 20 Tage der 100 Tag Tradition dürfen Unwürdige, welche den Bartvertrag nicht unterzeichnet haben, nachträglich den Bartvertrag unterzeichnen und die Historikerverantwortung
  aufnehmen. Dies muss aber vom Oberst-Bartvogt und mindestens einem weiteren Bartvogt genehmigt werden. Die genehmigenden Bartvögte müssen eine proportionale Strafe für den verspäteten Eintritt ernennen,
  welche der / die neue Vor-Würdige erfüllen muss. Wir schlagen einen halbseitigen Entschuldigungsbrief vor, der an alle Vor-Würdigen verteilt wird. Der / die nachträgliche Vor-Würdige erhält die Seiten des "<C-Handbuch unserer Geschichte">
  und die Erkennungshülle von seinem Bartvogt. Die Seiten stammen aus der Verteilungssicherheitsmarge \refS{H.verteilungsmarge}.

  \Clause[title={Erkennungs-, Schutz-, Vorweis-, Wissensaufgabe}] \label{H.aufgaben}
  Die Vor-Würdigen verpflichten sich, vier Aufgaben zu erfüllen.

  \SubClause[title={Erkennungsaufgabe}]\label{H.erkennungsaufgabe}
  Die Vor-Würdigen verpflichten sich, die Erkennungshülle (oder äquivalentes \refS{aequivalentes}) bei jedem Aufenthalt am Technikum während der
  Barttraditionsdauer sichtbar zu tragen. Sie dient als Erkennungszeichen der Vor-Würdigen.

  Alle Vor-Würdigen sind ermächtigt und ermutigt, das Erkennungszeichen zu kontrollieren und bei Missachtung der Erkennungsaufgabe ihren zugeteilten Bartvogt darüber zu informieren.

  Es ist im Ermessen der Bartvögte, ein äquivalentes Erkennungszeichen bilateral mit Interessenten zu vereinbaren.\label{aequivalentes}

  Wird ein:e Vor-Würdige:r von einer Person von ausserhalb der Würdigengemeinschaft, ab hier als ``Zivilist:in'' bezeichnet, auf ihr Erkennungszeichen angesprochen, so muss Sie eine Kombination
  der \refS{H.vorweisaufgabe} und \refS{H.wissensaufgabe} erfüllen und die Barttradition und Historikerverantwortung \refS{H.verantwortung} erklären.

  \SubClause[title={Schutzaufgabe}]\label{H.schutzaufgabe}
  Die Vor-Würdigen verpflichten sich, die Seiten des "<C-Handbuch unserer Geschichte"> vor jeglicher Beschädigung zu schützen.

  \SubClause[title={Vorweisaufgabe}]\label{H.vorweisaufgabe}
  Die Vor-Würdigen verpflichten sich, die Seiten des "<C-Handbuch unserer Geschichte"> auf Anfrage vorzuweisen. Dies kann durch Vorweisen der originalen
  Seiten oder einer digitalen Kopie erfolgen.

  \SubClause[title={Wissensaufgabe}]\label{H.wissensaufgabe}
  Die Vor-Würdigen verpflichten sich, den Inhalt ihrer Seiten zu studieren und zu verstehen. Sie müssen in der Lage sein, die Informationen
  auf Anfrage zusammenzufassen und zu erklären. Ist die anfragende Person zufrieden mit der Antwort, so wird sie vom Vor-Würdigen gebeten,
  ihre Unterschrift das aktuelle Datum auf die Rückseite der Erkennungshülle (NICHT auf dem "<C-Handbuch unserer Geschichte">) zu hinterlassen.

  Die Wissensaufgabe gilt nur als erfüllt, wenn mindestens eine Unterschrift gesammelt wurde. Die anfragende Person muss nicht Teil der Würdigengemeinschaft sein.

  Die Bartvögte sind ermutigt, die Vor-Würdigen unter sich zu verteilen und zu prüfen.

  Der / die Vor-Würdige mit den meisten einzigartigen Unterschriften wird mit einem Getränkegutschein von einem noch zu definierenden Wert, oder Äquivalentes
  belohnt. Haben mehrere Vor-Würdige die gleiche Anzahl an Unterschriften, so wird der Gutschein unter diesen geteilt.

  \SubClause[title={Dauer}]
  Die Schutzaufgabe \refS{H.schutzaufgabe} gilt von Annahme der Seiten des "<C-Handbuch unserer Geschichte"> bis zur Einsammlung \refS{H.einsammlung}.

  Die Erkennungs- \refS{H.erkennungsaufgabe}, Vorweis- \refS{H.vorweisaufgabe} und Wissensaufgabe \refS{H.wissensaufgabe} gelten von Annahme der Seiten
  des "<C-Handbuch unserer Geschichte"> bis zum BartAb an der Nacht der Technik am 4. Juli 2025.

  \Clause[title={Historikerverantwortung}]\label{H.verantwortung}
  Die in \refS{H.aufgaben} aufgezählten Aufgaben sind die Pflichten aller Vor-Würdigen und sind nicht übertragbar und definieren die Historikerverantwortung.
  Bei erfolgreicher Erfüllung der Historikerverantwortung bis zum Tag der Diplomfeier wird der / die Vor-Würdige offiziell als Würdige:n anerkannt und darf sich
  als Mitglied der Würdigengemeinschaft bezeichnen. Die Annerkennung wird durch eine Unterschrift der / des Würdigen auf einer seiner Seiten des "<C-Handbuch unserer Geschichte">
  an der Diplomfeier vor der Einsammlung \refS{H.einsammlung} bestätigt. Dafür stehen die Ränder und Rückseite der Seiten zur Verfügung. Die Würdigen sind
  gebeten auf die nächsten Generationen zu achten und den verwendeten Platz für ihre Unterschrift zu beschränken.

  \SubClause[title={Zusatztitel}]
  Es ist möglich bei Erfüllung einer Zusatzverantwortung, z.B. \refS{B.verantwortung} oder \refS{ersatz}, einen Zusatztitel zu erhalten. Dieser Titel ist explizit ein optionaler
  Zusatztitel und ist nicht als Ersatz zum Titel "<Würdige:r"> zu verstehen. Alle Vor-Würdigen, die die Historikerverantwortung \refS{H.verantwortung} erfüllten und ihre Unterschrift
  im "<C-Hanbuch unserer Geschichte"> hinzufügten, tragen den lebenslangen Titel "<Würdige:r">.

  \Clause[title={Unwürdige}]
  Als Unwürdig gelten alle Klassenmitglieder, welche den Bartvertrag nicht unterzeichnet haben. Diese gehen der Geschichte verloren. In 100 Jahren bleibt das
  "<C-Handbuch unserer Geschichte"> weiter bestehen aber Unwürdige nicht.

  Bei Missachtung der Historikerverantwortung \refS{H.verantwortung} wird der / die Vor-Würdige von der Würdigengemeinschaft ausgeschlossen und ist für die Ewigkeit als
  Un-Würdige:r bezeichnet.

  \Clause[title={Anti-Würdige}]
  Die Missachtung der Schutzaufgabe \refS{H.schutzaufgabe} gilt als schwerwiegend, da sie die Integrität des "<C-Handbuch unserer Geschichte"> und die
  damit verbundene Kontinuität unserer Gemeinde gefährdet. Bei Verlust oder grober Beschädigung der Seiten des "<C-Handbuch unserer Geschichte"> wird der / die Vor-Würdige sogar als
  Anti-Würdige:r bezeichnet und die Schmach wird auf seine ganze Blutlinie übertragen.

  \Clause[title={Einsammlung}]\label{H.einsammlung}
  Die Seiten des "<C-Handbuch unserer Geschichte"> werden an der Diplomfeier von einem designierten Bartvogt eingesammelt und an die IT-Vorfahren übergeben.

  Da für die Studiengänge der Zürcher Hochschule für Angewandte Wissenschaften (ZHAW) getrennte Diplomfeiern stattfinden, wird ein:e Würdige:r erwählt, der die gesammelten
  Seiten einer Diplomfeier den Bartvögten der nächsten Diplomfeier übergibt. Diese:r Würdige:r erhält den ehrenvollen Titel "<Würdige:r der Vereinigung der Diplomfeiern"> und darf
  diesen Titel mit Jahr unter seine Unterschrift im "<C-Handbuch unserer Geschichte"> notieren.

  Nach der Vereinigung des "<C-Handbuch unserer Geschichte"> wird ein:e Würdige:r ernannt, der / die es an den FAI übergibt, welcher die Archivistenaufgabe \refS{H.archivistenaufgabe} übernimmt.

  Die Erkennungshüllen werden spätestens an der Diplomfeier abgegeben, an die Bartvögte übergeben und an den / die IT-Vorfahrin:en weitergeleitet.

  Die unterschriebene Titelseite der Erkennungshülle bleibt im Besitz der Würdigen und dient als Beweis der Erfüllung der Historikerverantwortung.

  \Clause[title={Archivistenaufgabe}]\label{H.archivistenaufgabe}
  Der FAI verpflichtet sich, das Original des "<C-Handbuch unserer Geschichte"> sowie auch digitalisierte Kopien zu archivieren und während bis zur nächsten letzten Rasur
  zu pflegen. Der / die IT-Vorfahre:in ist verantwortlich für die Koordination und die Übergabe an den Bartvogt der nächsten Generation.

  Der IT-Alumnieverein soll auch jeden Gesamtbartvertrag archivieren, entweder im Original oder in digitaler Form. Der IT-Vorfahre und Oberst-Bartvögt sind verantwortlich für die Koordination
  und Übergabe des Gesamtbartvertrags an den IT-Alumniverein.

  Ist das "<C-Handbuch unserer Geschichte"> mit Unterschriften der Würdigen voll, so wird es als "<C-Handbuches unserer Geschichte V1.0"> getauft und eine neue Kopie erstellt.

  Die Versionsnummer wird auf der Titelseite des "<C-Handbuch unserer Geschichte"> notiert und die alte Version wird im Archiv des FAI aufbewahrt.

  Mit jeder neuen Version wird die Major-Version Nummer inkrementiert.

  \Clause[title={Fittnessklausel}]
  Wenn ein:e Vor-Würdige:r dem Oberst-Bartvogt eine Sequenz von acht Pistolsquats pro Bein vorführt, wird er / sie automatisch als Würdige:n anerkannt und darf seine / ihre Unterschrift
  dem "<C-Handbuch unserer Geschichte"> mit dem Titel "<Würdiger der Fitness"> hinzufügen und seine Seiten an den Oberst-Bartvogt übergeben. Er / Sie ist von allen weiteren
  Aufgaben und Pflichten befreit.

\end{contract}

\pagebreak
\section{Bartverantwortung}
\begin{contract}
  \Clause[title={Bartverantwortung}]\label{B.verantwortung}
  Zusätzlich zu der Historikerverantwortung \refS{H.verantwortung} dürfen Vor-Würdige auch die Bartverantwortung übernehmen. Die Bartverantwortung ist
  optional.

  \Clause[title={Letzte Rasur}]\label{B.lastrasur}
  Jede:r Vor-Würdige, der / die zusätzlich die Bartverantwortung übernehmen will, hat sich am in \refS{gueltigkeit} genannten Datum um 13:00 Uhr
  beim Fischermädchenbrunnen in der Steinberggasse einzufinden.
  Die Rasur findet gemeinsam statt.

  \SubClause[title={Durchführung der Rasur}]\label{B.durchfuehrung}
  Um die Tradition zu ehren, sind ausschliesslich Nassrasuren gestattet. Die Verwendung von
  elektronischen Gerätschaften und sonstiger Hexereien ist strengstens untersagt. Erforderlich für eine
  ehrenvolle Rasur ist die vollständige Entfernung der Gesichtsbehaarung, ausgenommen der
  Augenbrauen. Die Bartvögte nehmen sich der Begutachtung der Rasur jedes einzelnen Mitglieds
  an.
  \SubClause[title={Anschliessende Kehlenbefeuchtung}]
  Am Ende der Rasur sind alle Vor-Würdigen eingeladen, vor Ort ein kühles Bier einzunehmen.

  \SubClause[title={Rasurverzug}]
  Ist einem ehrenvollen Mitglied die Rasur zur vorgesehenen Zeit und Stelle nicht möglich,
  ist es diesem gestattet, die Rasur bis zum 31. März 2025 nachzuholen, sofern ein triftiger
  Grund vorgewiesen werden kann. Die Bartvögte entscheiden über die Zulässigkeit des Grundes
  und überprüfen die Rasur, wenn möglich persönlich oder alternativ durch einen
  Microsoft Teams-Call mit eingeschalteter Webcam.

  \SubClause[title={Glaubensklausel (Shamess Klausel)}]
  Sollte die Glattrasur aufgrund glaubenstechnischer Regelungen untersagt sein, meldet sich
  der betroffene ehrenvolle Vor-Würdige bei den Bartvögten, um die Rasur-Einschränkung zu
  besprechen. Sollten die Bartvögte mit der Mindestlänge an Bart einverstanden sein,
  kann das ehrenvolle Mitglied, abgesehen der eingeschränkten Rasuren, normal
  an der Bartverantwortung teilnehmen.

  \Clause[title={Bartvorschriften}]
  \SubClause[title={Definition Bart}]
  Als Bart wird eine ununterbrochene Linie mit körpereigenen Gesichtshaaren zwischen den beiden Ohren,
  welche über Kinn und Oberlippe verläuft und bis zum Halsansatz reicht, verstanden. Eine Bartaneignung
  durch Fremdhilfe wie Haartransplantation, Bartimitationen, tätowierte oder aufgemalte Bärte,
  Skalpe oder ähnliches sowie die Verwendung von Haarwuchsmittel sind untersagt und gelten als Vertragsbruch.

  \SubClause[title={Bartpflege}]\label{B.bartpflege}
  Die Würde des Bartes ist durch regelmässige Pflege in Ehren zu halten. Ungepflegte, verfilzte oder gar
  verlauste Bärte, sind der langen Tradition nicht würdig. Das Vernachlässigen der adäquaten Bartpflege
  wird von den Bartvögten geahndet. Vor-Würdige, die sich dessen schuldig gemacht haben, erwartet die
  Einnahme eines Rachenputzers \refS{B.rachenputzer} oder den Ausschank eines Bieres an zwei beliebige Vor-Würdige der Klasse.
  Zudem muss die würdige Person die regelmässige Pflege des Bartes innert 24 Stunden aufnehmen. Bei
  weiteren Verstössen erhöht sich die Anzahl der Rachenputzer oder Bierausschänke pro Verstoss um den
  Faktor zwei. Das Trimmen von Kontur und die Stutzung des Schnauzes, um eine einwandfreie
  Lebensmittelaufnahme zu gewährleisten ist erlaubt. Regelmässiges Waschen, Föhnen und Kämmen wird empfohlen.

  \SubClause[title={Rachenputzer}]\label{B.rachenputzer}
  Was genau als Rachenputzer gilt, soll den Bartvögten überlassen werden. Grundsätzlich versteht sich unter
  Rachenputzer ein Getränk alkoholischer oder nichtalkoholischer Natur. Ein Rachenputzer soll als Bestrafung fungieren,
  wobei der Wiedereingliederungsgedanke in die würdige Gesellschaft im Vordergrund stehen soll.
  Die Ehre des würdigen Vor-Würdigen darf darum nicht mehr als nötig verletzt werden.

  \SubClause[title={Schneiden}]
  Ab dem Zeitpunkt der letzten Rasur gemäss \refS{B.lastrasur} darf dem Bartwuchs nicht entgegengewirkt werden.
  Jedes Kürzen der Bartbehaarung, ausgenommen ist der Fall\refS{B.bartpflege}, ist untersagt und gilt als Bruch der Bartverantwortung.

  \Clause[title={Bartprüfung}]\label{B.bartpruefung}
  Die Bartverantwortung gilt als abgelegt, wenn die Bartprüfung bestanden wurde. Die Bartprüfung wird volgendermassen definiert:

  \SubClause[title={Durchführung}]
  Um die Würde der Bartträgerschaft endgültig zu festigen, müssen sich alle Vor-Würdigen, die die Bartverantwortung aufgenommen haben,
  einer Bartprüfung unterziehen, welche am Tag vor der Nacht der Technik stattfindet. Die Prüfung wird von den
  Bartvögten geleitet und durchgeführt. Der Durchführungsort ist ebenfalls der Fischermädchenbrunnen in der
  Steinberggasse.

  \SubClause[title={Handlung}]
  Traditionsgemäss steckt sich jede Person, die die Prüfung ablegen möchte, einen Bleistift in den Bart. Die
  Prüfung gilt als bestanden, wenn der Bleistift nach einem Sprung vom Brunnenrand auf den Boden noch an
  derselben Stelle sitzt. Die Prüfung darf dreimal wiederholt werden. Wird die Prüfung auch nach dem dritten Versuch nicht bestanden,
  so zählt dieses Mitglied zu den Gescheiterten \refS{B.gescheiterte}.

  \SubClause[title={Zustand des Bartes}]
  Zum Zeitpunkt der Prüfung muss sich der Bart in sauberem Zustand befinden. Klebrige Rückstände von Getränken,
  Speisen, Gel, Wachs oder Ähnlichem müssen vollständig entfernt worden sein.

  \SubClause[title={Prüfungsaufsicht}]
  Die Prüfungsaufsicht obliegt den Bartvögten.

  \SubClause[title={Prüfung der Bartvögte}]
  Um die Unabhängigkeit der Prüfungsbewertung zu gewährleisten, dürfen die Bartvögte ihre eigene Prüfung nicht selbst bewerten.
  Zu diesem Zweck werden aus der versammelten Gemeinschaft zwei Interimsvögte ausgewählt, die die Prüfung
  der Bartvögte beaufsichtigen. Diese werden nach dem Zufallsprinzip ausgewählt. Zur Auswahl dürfen Würfel,
  Schere-Stein-Papier, A zellä Böllä schelä oder Ähnliches zum Einsatz kommen. Die Bartvögte legen die
  Prüfung als erste ab.

  \Clause[title={BartAb}]
  Nach Ablegen der Bartprüfung und der Durchführung des Frackumzugs darf sich jeder Vor-Würdige, der die Bartverantwortung aufgenommen
  hat am Bart Ab Event rasieren lassen, wobei dieselben Bestimmungen des \refS{B.durchfuehrung} gültig sind.

  \Clause[title={Gescheiterte}]\label{B.gescheiterte}
  Als Gescheiterte gelten Vor-Würdige, welche trotz ihres guten Willens die Prüfung gemäss \refS{B.bartpruefung}
  aufgrund ihres spärlichen Bartwuchses nicht bestanden haben.

  \Clause[title={Titel}]\label{B.barttitel}
  Eine Erfüllung der Bartverantwortung gemäss \refS{B.bartpruefung} berechtigt den Vor-Würdigen bei Erfüllung der Historikerverantwortung \refS{H.verantwortung}
  und der Erhebung zum Würdigen, zum Titel "<Bartwürdiger">. Damit darf dessen Unterschrift im "<C-Handbuch unserer Geschichte"> mit dem Titel "<Bartwürdiger"> oder eine andere
  Beschmückung versehen werden.

\end{contract}

\pagebreak
\section{Ersatzverantwortungen}
\begin{contract}
  \Clause[title={Ersatzverantwortung}]\label{ersatz}
  Anstelle der Bartverantwortung dürfen einzelne Klassen eine Ersatzverantwortung definieren. Diese Verantwortungen und dessen Strafen werden klassenintern
  definiert und geregelt.

  Die Ersatzverantwortung befreit die Vor-Würdigen NICHT von der Historikerverantwortung \refS{H.verantwortung}.

  Die Ersatzverantwortung gilt als äquivalent zur Bartverantwortung und berechtigt bei Erfüllung einen äquivalenten Titel zu \refS{B.barttitel} bei der Unterzeichnung
  im "<C-Handbuch unserer Geschichte"> nach Erfüllung der Historikerverantwortung \refS{H.verantwortung}. Der Titel muss im Vertrag definiert sein.

  \Clause[title={Klasse IT22a WIN}]
  Die Klasse IT22a WIN verzichtet bedauerlicherweise auf Teilnahme an dem Gesamtbartvertrag.

  \Clause[title={Klasse IT22a ZH}]\label{KlasseIT22aZH.ersatz}
  Die Klasse IT22a ZH hat sich für eine Ersatzverantwortung entschieden und definiert diese in den folgenden Unterartikeln.
  \SubClause[title={Tragepflicht von Utensilien/Kleidungsstücken in Neonfarben}]\label{KlasseIT22aZH.clothing}
  Alle Teilnehmenden sind verpflichtet, an jedem Tag innerhalb des in \refS{gueltigkeit} definierten Zeitraums ein Utensil oder Kleidungsstück in einer der folgenden Farben sichtbar zu tragen: Neongrün, Neongelb, Neonorange oder Neonpink.

  Jeder Teilnehmende ist berechtigt, an jedem der 100 festgelegten Tage von einem anderen Teilnehmenden den Nachweis über das Tragen eines entsprechenden Utensils oder Kleidungsstücks in Form eines Bildes zu verlangen.

  Wird festgestellt, dass ein Teilnehmender dieser Verpflichtung nicht nachkommt, so ist dieser verpflichtet, unverzüglich ein alkoholisches Getränk in der Menge von 0,33 Litern (Bier) in einem Zuge („Ex-Trinken“) zu
  konsumieren und den Nachweis hierüber in Form eines Videoaufzeichnungsbeweises an die übrigen Teilnehmenden zu übermitteln.

  \SubClause[title={Transportpflicht persönlicher Gegenstände an Freitagen}]\label{KlasseIT22aZH.transport}
  Persönliche Gegenstände sind an jedem Freitag während der Vorlesungszeit in einem Behältnis zu transportieren, das weder einen Rucksack darstellt oder diesem ähnelt noch eine Handtasche darstellt oder dieser ähnelt.

  Jedes Behältnis darf höchstens an zwei aufeinanderfolgenden Freitagen genutzt werden. Nach Ablauf dieses Zeitraums ist ein anderes Behältnis zu verwenden.

  Wird die Teilnahme am Unterricht an einem Freitag unterlassen, so tritt eine Ersatzverpflichtung ein:
  \begin{enumerate}[label=(\alph*)]
    \item Der betreffende Teilnehmende ist verpflichtet, zwei alkoholische Getränke in der Menge von jeweils 0,33 Litern (Bier) zu konsumieren.
    \item Das erste Getränk ist spätestens um 08:00 Uhr, das zweite Getränk spätestens um 09:00 Uhr zu konsumieren.
    \item Der Nachweis hierüber ist durch die Anfertigung und Übermittlung entsprechender Videoaufzeichnungen an die übrigen Teilnehmenden zu erbringen.
  \end{enumerate}

  Verstöße gegen die in \refS{KlasseIT22aZH.clothing} und \refS{KlasseIT22aZH.transport} genannten Bestimmungen sind bis spätestens zum darauffolgenden Donnerstag nachzuholen.

  \SubClause[title={Verleihung des Titels "<Illuminierter Ehrenwürdiger">}]
  Ein Teilnehmender, der während der gesamten in \refS{gueltigkeit} definierten Zeitspanne ausnahmslos die in \refS{KlasseIT22aZH.ersatz} festgelegten Regelungen einhält,
  erwirbt mit Ablauf des Zeitraums das Recht auf die Verleihung des Ehrentitels "<Illuminierter Ehrenwürdiger">.

  Die Verleihung dieses Titels erfolgt automatisch, sofern keine Verstöße gegen die in \refS{KlasseIT22aZH.ersatz} niedergelegten Pflichten festgestellt wurden.

  Etwaige Streitfälle oder Unklarheiten hinsichtlich der Erfüllung der Voraussetzungen werden durch eine Mehrheitsentscheidung der übrigen Teilnehmenden geklärt.

  \Clause[title={Klasse IT21ta WIN}]
  Die Klasse IT21ta WIN verzichtet bedauerlicherweise auf Teilnahme an dem Gesamtbartvertrag.

  \Clause[title={Klasse IT21tb WIN}]
  Die Klasse IT21tb WIN verzichtet auf eine Ersatzverantwortung.
\end{contract}


\pagebreak
\section{Unterschriften}
\subsection*{IT22a ZH}
\vspace{50pt}
\noindent\rule{7cm}{.4pt}\hfill\rule{7cm}{.4pt}\par
\noindent Ueltschi, Damian \hfill Willi, Leandra

\subsection*{IT21tb WIN}
\vspace{50pt}
\noindent\rule{7cm}{.4pt}\hfill\rule{7cm}{.4pt}\par
\noindent Meili, Lukas \hfill Müller, Dominik

\vspace{50pt}
\noindent\rule{7cm}{.4pt}\hfill\rule{7cm}{.4pt}\par
\noindent Baumgartner, Noah \hfill Neuschwander, Cyrill

\vspace{50pt}
\noindent\rule{7cm}{.4pt}\hfill\rule{7cm}{.4pt}\par
\noindent Wäspe, Lara \hfill Risitc, Dragana

\vspace{50pt}
\noindent\rule{7cm}{.4pt}\hfill\rule{7cm}{.4pt}\par
\noindent Schlauri, Noah \hfill Aubry, Dario

\vspace{50pt}
\noindent\rule{7cm}{.4pt}\hfill\rule{7cm}{.4pt}\par
\noindent Ferrari, Leonardo \hfill Elhayawan, Karim

\vspace{50pt}
\noindent\rule{7cm}{.4pt}\hfill\rule{7cm}{.4pt}\par
\noindent Stumpf, Simon \hfill

\vspace{50pt}
\noindent\rule{7cm}{.4pt}\hfill\rule{7cm}{.4pt}\par
\noindent \hfill

\vspace{50pt}
\noindent\rule{7cm}{.4pt}\hfill\rule{7cm}{.4pt}\par
\noindent  \hfill

\vspace{50pt}
\noindent\rule{7cm}{.4pt}\hfill\rule{7cm}{.4pt}\par
\noindent  \hfill

\end{document}